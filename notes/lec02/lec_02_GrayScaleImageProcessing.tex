% !TEX root=./../maha-dip-notes.tex
\chapter{Gray Scale Image Processing}

\section{Point Operations}

Point operations are image transformations where each output pixel is computed solely from its corresponding input pixel, without regard for neighboring values.

\dfn{Point Operation}{A transformation applied to each pixel individually, $J(x,y) = T[I(x, y)]$, where $I(x, y)$ is the input image and $T$ is a point-wise mapping function.}

These operations can be linear or non-linear, and are different from local operators/spatial filters (i.e., those that consider neighboring pixels such as ERODE, DILATE, etc.).

No explicit neighborhood information is used in point operations, and no explicit modifications based on spatial context. however, they can still be influenced by global image properties (e.g., overall brightness, contrast).



\subsection{Image Offset (Brightness Adjustment)}
The simplest point operation is the offset, where a constant value $c$ is added to all pixel intensities:
$$
J(x, y) = I(x, y) + c
$$
This increases (or decreases) the brightness of the image uniformly.
\begin{itemize}
    \item If $c > 0$, the image becomes brighter.
    \item If $c < 0$, the image becomes darker.
\end{itemize}

\ex{Image Offset Application}{
    Suppose $I_1,I_2,\dots, I_N$ are $N$ images of the same scene taken under different lighting conditions (different exposures). If we wish to equalise all their average intensities to $\frac{K}{2}$ (where K is number of intensity levels), a point operation can be used to normalize brightness across these images by applying an offset to each image:
    $$J_i(x, y) = I_i(x, y) + c_i$$
    where $c_i$ is the offset computed as:
    $$c_i = \frac{K}{2} - \text{mean}(I_i)$$
}

\subsection{Image Scaling (Contrast Adjustment)}
Image scaling multiplies all pixel intensities by a constant factor $a$:
$$
J(x, y) = a \cdot I(x, y)
$$
A scaling factor $a > 1$ increases contrast, while $0 < a < 1$ darkens the image.


\subsection{Offset and Scaling}
Combining offset and scaling operations allows for more flexible image adjustments. The combined operation can be expressed as:
$$
J(x, y) = a \cdot I(x, y) + c
$$
Where $a$ is the scaling factor and $c$ is the offset. This allows for both contrast adjustment (via scaling) and brightness adjustment (via offset) in a single operation.

\ex{Negative Image}{
    The negative of an image is obtained by inverting the pixel values by choosing :
    \begin{itemize}
        \item image offset  $c = K - 1$, Where $K$ is the number of intensity levels.
        \item image scaling $a = -1$
    \end{itemize}
    Thus,
    $$
    J(x, y) = K - 1 - I(x, y)
    $$
    This operation effectively reverses the brightness levels, making dark areas light and vice versa.
}

\ex{Full Scale Contrast stretch}{
    Let A \& B be the minimum and maximum intensity values in the image respectively. We wish to stretch the contrast to full scale $[0, K-1]$. This can be achieved from $aA + c = 0$ and $aB + c = K - 1$, we can solve for $a$ and $c$, we get:
    \begin{itemize}
        \item image scaling $a = \frac{K-1}{B-A}$
        \item image offset  $c = -\frac{K-1}{B-A} \cdot A$
    \end{itemize}
    Thus,
    \begin{align*}
    J(x, y) &= \frac{K-1}{B-A} \cdot I(x, y) - \frac{K-1}{B-A} \cdot A\\
    &= \frac{K-1}{B-A} \cdot (I(x, y) - A)
    \end{align*}
    This allows us to stretch the pixel values to cover the full range of intensities, hence enhances the overall contrast of the image.
}

\subsection{Non-linear Point Operations}
Non-linear point operations are techniques that apply non-linear transformations to the pixel values of an image. These operations can enhance certain features or suppress others, depending on the specific transformation applied.

\paragraph{Common Examples}
\begin{itemize}
    
    \item \textbf{Gamma Correction:}
        $$
        J(x, y) = c \cdot I(x, y)^\gamma \qquad \gamma > 0
        $$
        Where $c$ is a scaling factor and $\gamma$ is the gamma value.
        \begin{itemize}
            \item $0 < \gamma < 1$ : Enhances dark regions (gamma correction).
            \item $\gamma > 1$ : Enhances bright regions (inverse gamma correction).
        \end{itemize}

        \begin{center}
        \begin{tikzpicture}
            \draw[->] (0,0) -- (5,0) node[right] {Input Intensity};
            \draw[->] (0,0) -- (0,5) node[above] {Output};
            \draw[domain=0:5,smooth,variable=\x,blue,thick] plot (\x,{pow(\x/5,0.5)*5});
            \draw[domain=0:5,smooth,variable=\x,red,dashed] plot (\x,{pow(\x/5,2)*5});
            % Two gamma curves: gamma=0.5 (blue, boost), gamma=2 (red, dampen)
        \end{tikzpicture}
        \\
        \emph{Gamma Correction ( Blue : $\gamma = 0.5 \rightarrow I^{\frac{1}{2}}$  , Red : $\gamma = 2 \rightarrow I^{2}$ )}
        \end{center}

    \item \textbf{Sigmoid Correction:}
        $$
        J(x, y) = \frac{K}{1 + \exp(-\beta [I(x, y) - \alpha])}
        $$
        Where $K$ is maximal intensity, $\alpha$ the mid-point, and $\beta$ the slope. Useful for contrast enhancement around a chosen intensity.

        \begin{center}
        \begin{tikzpicture}
            \draw[->] (0,0) -- (5,0) node[right] {Input Intensity};
            \draw[->] (0,0) -- (0,5) node[above] {Output};
            \draw[domain=0:5,smooth,variable=\x,blue,thick] plot (\x,{5/(1 + exp(-9*(\x/5-0.5)))});
            % Sigmoid curve: midpoint at 0.5, slope=9
        \end{tikzpicture}
        \\
        \emph{Sigmoid Correction Curve}
        \end{center}


    \item \textbf{Other Non-linear Operations:}
        \begin{itemize}
            \item \emph{Logarithmic Correction:} $g(x, y) = c \cdot \log(1 + f(x, y))$
            \item \emph{Exponential Correction:} $g(x, y) = a \cdot \exp(f(x, y)/b)$
            \item \emph{Piece-wise Linear Transformations:} used for specific intensity mapping (e.g., contrast stretching).
        \end{itemize}
\end{itemize}



\nt{Non-linear point operations are often employed to address unique characteristics of image acquisition (camera response, display systems) or to prepare images for further processing steps like thresholding or segmentation.}



\section{Histogram Equalization}

Histogram equalization is a method to adjust image intensities to enhance contrast. The histogram of an image is "flattened" (equalized), redistributing pixel values as uniformly as possible over the intensity range.

\dfn{Histogram Equalization}{A point-wise image transformation that remaps the intensity distribution so that the output histogram is (approximately) uniform. This increases the dynamic range and enhances global contrast.}

\noindent Given a gray scale image with intensities $r$ in $[0, K-1]$ and histogram $h(r)$, histogram equalization computes a transformation $HE$ as the cumulative distribution function (CDF):
$$
HE(r) = \sum_{k=0}^{r} \frac{h(k)}{MN}
$$
where $MN$ is the total number of pixels.

\dfn{Histogram Equalization Transformation}{
The transformation $s = HE(r)$ maps each input intensity $r$ to an output intensity $s$ so that $s$ is approximately uniformly distributed over $[0, K-1]$.
}
